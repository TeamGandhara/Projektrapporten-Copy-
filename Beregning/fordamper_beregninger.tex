\documentclass[../Hovedrapport.tex]{subfiles}

\mathindent 0.0in
\usepackage[ansinew]{inputenc}
\usepackage{times}
\usepackage{graphicx}
\usepackage{color}
\usepackage{textcomp}
\definecolor{silver}{rgb}{0.75,0.75,0.75}
\definecolor{gray}{rgb}{0.5,0.5,0.5}
\definecolor{aqua}{rgb}{0.5,1,1}
\definecolor{navy}{rgb}{0.0,0.0,0.5}
\definecolor{orange}{rgb}{1.0,0.5,0.0}
\definecolor{teal}{rgb}{0.25,0.5,0.5}
\definecolor{olive}{rgb}{0.5,0.5,0.0}
\definecolor{purple}{rgb}{0.5,0.0,0.5}
\definecolor{brown}{rgb}{0.5,0.25,0.0}
\definecolor{fuchsia}{rgb}{1.0,0.5,1.0}
\definecolor{buff}{rgb}{1.0,0.94,0.80}
\definecolor{lime}{rgb}{0.5,1.0,0.0}
\setlength{\headsep}{-0,6in}
\setlength{\textheight}{9in}
\setlength{\footskip}{1.0 in}
\setlength{\oddsidemargin}{-0,2in}
\setlength{\evensidemargin}{-0,2in}
\setlength{\textwidth}{6,8in}
\usepackage{longtable}
\def\headline#1{\hbox to \hsize{\hrulefill\quad\lower .3em\hbox{#1}\quad\hrulefill}}
\newcommand{\abs}[1]{\left|#1\right|}
\newcommand{\F}[1]{\mbox{$#1$}}
\newcommand{\K}[1]{\mbox{\sf#1\ \ \mit}}
\newcommand{\KS}[1]{\mbox{\sf\ \ #1\ \ \mit}}
\newcommand{\SC}[1]{\mbox{`#1'}\  }
\newcommand{\V}[1]{\mbox{$ #1 $}}
\newcommand{\I}{\mbox{\hspace{0.20in}}}
\newcommand{\temperature}{\mathrm{T}}
\newcommand{\pressure}{\mathrm{P}}
\newcommand{\volume}{\mathrm{v}}
\newcommand{\density}{\mathrm{\rho}}
\newcommand{\intenergy}{\mathrm{u}}
\newcommand{\enthalpy}{\mathrm{h}}
\newcommand{\entropy}{\mathrm{s}}
\newcommand{\molarmass}{\mathrm{MW}}
\newcommand{\enthalpyfusion}{\mathrm{\Delta h_{fusion}}}
\newcommand{\quality}{\mathrm{x}}
\newcommand{\viscosity}{\mathrm{\mu}}
\newcommand{\conductivity}{\mathrm{k}}
\newcommand{\prandtl}{\mathrm{P_r}}
\newcommand{\cp}{\mathrm{c_p}}
\newcommand{\cv}{\mathrm{c_v}}
\newcommand{\specheat}{\mathrm{c_p}}
\newcommand{\soundspeed}{\mathrm{c}}
\newcommand{\wetbulb}{\mathrm{wb}}
\newcommand{\humrat}{\mathrm{\omega}}
\newcommand{\acentricfactor}{\mathrm{\omega}}
\newcommand{\relhum}{\mathrm{\phi}}
\newcommand{\dewpoint}{\mathrm{DP}}
\newcommand{\volexpcoef}{\mathrm{\beta}}
\newcommand{\compressibilityfactor}{\mathrm{Z}}
\newcommand{\surfacetension}{\mathrm{\gamma}}
\newcommand{\tcrit}{\mathrm{T_{crit}}}
\newcommand{\pcrit}{\mathrm{P_{crit}}}
\newcommand{\vcrit}{\mathrm{v_{crit}}}
\newcommand{\ttriple}{\mathrm{T_{triple}}}
\newcommand{\fugacity}{\mathrm{fugacity}}
\newcommand{\tsat}{\mathrm{T_{sat}}}
\newcommand{\psat}{\mathrm{P_{sat}}}
\newcommand{\eklj}{\mathrm{ek_{LJ}}}
\newcommand{\sigmalj}{\mathrm{\sigma_{LJ}}}
\newcommand{\isentropicexponent}{\mathrm{k_{s}}}
\newcommand{\thermaldiffusivity}{\mathrm{\alpha}}
\newcommand{\kinematicviscosity}{\mathrm{\nu}}
\newcommand{\isothermalcompress}{\mathrm{K_{T}}}


\begin{document}

MODEL TIL BEREGNINGER På FORDAMPER

\subsection*{Beregninger}

\vspace{0.10in}
\noindent
\rm Beregning p\aa eksisterende fordamper\newline
 \newline
 Fremgangsm\aade:\newline
 \newline
For at beregne Varmegennemgangstal, U og kuldeydelsen, $\phi$$_{V}$, skal nedenst\aaende fremgangsm\aade benyttes.\newline
Hvert punkt kan først udføres n\aar alle dets underpunkter er udført.\newline
 \newline
 \newline
	1 - Beregn den logaritmiske temperaturdifferens, ${\delta t}$$_{m}$\newline
 \newline
	      1.1 - til dette benyttes formlen for middeltemperaturdifferens fra Aages noter om lamelvarmevekslere. [den kan findes et andet sted]\newline
 \newline
 \newline
	2 - Beregn den totale Varmeovergangsmodstand, SigmaR\newline
 \newline
	      2.1 - Beregn Varmeovergangsmodstand, R$_{ou}$, ved konvektion p\aa lameller og rør\newline
 \newline
	            2.1.1 - bestem varmeovergangstal p\aa lameller og rør til luft, $\alpha$$_{u}$, ud fra Th. E. Schmidts modelligning for Nusselts tal p\aa ribberørbundt\newline
	                  2.1.1.1 - beregn Nusselts tal ved brug af 2. del af modelligningen\newline
	                  2.1.1.2 - beregn $\alpha$$_{u}$ ud fra nusselts tal og 1. del af modelligningen\newline
 \newline
	            2.1.2 - Bestem det ydre overfladeareal, A$_{u}$ (denne er allerede udregnet som A$_{t}$ i afsnit 2.1.1.1)\newline
 \newline
	            2.1.3 - Bestem finnevirkningsgraden, $\eta$$_{R}$\newline
	                  2.1.3.1 - Dette gøres som beskrevet i Danvak Grundbog s. 424\newline
 \newline
 \newline
	      2.2 - Beregn Varmeovergangsmodstand, R$_{oi}$, ved konvektion indvendigt i rør\newline
 \newline
 	            2.2.1 - Bestem varmeovergangstallet over rørets indervæg, $\alpha$$_{i.}$ Dette gøres ud fra Bo Pierres metode [køleteknik afsnit 6]\newline
	                  2.2.1.1 - Bestem nusselts tal med Bo Pierres metode\newline
	                  2.2.1.2 - Benyt denne nusselts tal til at bestemme varmeovergangstallet $\alpha$$_{i}$\newline
	 \newline
	            2.2.2 - Bestem rørets indre overfladeareal, A$_{i}$\newline
 \newline
 \newline
	      2.3 - Beregn Varmeovergangsmodstand i rørvæg, R$_{rør}$\newline
 \newline
	            2.3.1 - dette gøres med formlen for varmeledningsmodstand i rørvæg [noter om lamelvarmeveklser - eller termodynamik afsnit 9]\newline
 \newline
	      2.4 - læg de tre varmeovergangsmodstande sammen\newline
 \newline
$_{;;;;;;;;;;;;;;;;;;;;;;;;;;;;;;;;;;;;;;;;;;;;;;;;;;;;;;;;;;;;;;;;;;;;;;;;;;;;;;;;;;;;;;;;;;;;;;;;;;;;;;;;;;;;;;;;;;;;;;;;\newline}$
 \newline
Selve udregningerne:
\begin{verbatim}
$REFERENCE R134a IIR
\end{verbatim}  \begin{verbatim}
$UnitSystem SI K bar J mass deg
\end{verbatim}  \begin{equation}
\label{EES Eqn:1}
R\$ = \SC{R134a}		 
\mbox{\I Kølemiddel i systemet }
\end{equation}

\vspace{0.10in}
\noindent
\rm Temperaturer i systemet:
\begin{equation}
\label{EES Eqn:2}
t_{L;1} =  \left( 10 + 273,15 \right)    \   \left[ \rm K \right]	 
\mbox{\I Luftens temperatur før fordamper [C] (rumtemperatur)}
\end{equation}
\begin{equation}
\label{EES Eqn:3}
t_{L;2} =  \left( 6 + 273,15 \right)    \   \left[ \rm K \right]	 
\mbox{\I Luftens temperatur efter fordamper [C] i første omgang gættes denne.}
\end{equation}

\vspace{0.10in}
\noindent
\rm i slutningen af dokumenter gives der en mere korrekt værdi for udgangstemperaturen t$_{L;3}$

\vspace{0.10in}
\noindent
\rm der skal $\nu$ prøves frem indtil t$_{L;2}$ ~ t$_{L;3}$
\begin{equation}
\label{EES Eqn:4}
t_{R;1} =  \left(  \left( -8 \right)  + 273,15 \right)    \   \left[ \rm K \right]	 
\mbox{\I kølemidlets temperatur før fordamper [C] }
\end{equation}
\begin{equation}
\label{EES Eqn:5}
t_{R;2} = t_{R;1	} 
\mbox{\I der antages at kølemidlet har konstant temperatur gennem fordamperen}
\end{equation}
\begin{equation}
\label{EES Eqn:6}
\epsilon = 1	 
\mbox{\I Korrektionsfaktor for krydsstrøm. hvis der er konstant t på en af siderne, sættes epsilon til 1 [køleteknik s. 47 figur 2.13]}
\end{equation}
\begin{equation}
\label{EES Eqn:7}
t_{Lm} = \frac {t_{L;1}+t_{L;2}}{ 2	 } 
\mbox{\I Middeltemperatur for luft [K]}
\end{equation}
\begin{equation}
\label{EES Eqn:8}
\phi_{ønsk} = 773   \   \left[ \rm W \right]	 
\mbox{\I Ønsket kuldeydelse i fordamperen [W]}
\end{equation}

\vspace{0.10in}
\noindent
\rm Varmevekslergeometri:
\begin{equation}
\label{EES Eqn:9}
N_{serie} = 1	 
\mbox{\I Antal ens varmevekslere i serie bag hinanden }
\end{equation}

\vspace{0.10in}
\noindent
\rm rør
\begin{equation}
\label{EES Eqn:10}
L_{1rør} = 0,22   \   \left[ \rm m \right]	 
\mbox{\I Længdet af ét rør i varmeveksleren [m]}
\end{equation}
\begin{equation}
\label{EES Eqn:11}
N_{rør} = 24\cdot N_{serie	} 
\mbox{\I antal gange røret passerer gennem varmeveksler}
\end{equation}
\begin{equation}
\label{EES Eqn:12}
N_{rækker} = 3 \cdot  N_{serie	} 
\mbox{\I antal rørrækker i alt}
\end{equation}
\begin{equation}
\label{EES Eqn:13}
L_{rør} = L_{1rør} \cdot  N_{rør	} 
\mbox{\I total længde af rør [m]}
\end{equation}
\begin{equation}
\label{EES Eqn:14}
d_{u} = 8   \   \left[ \rm mm \right] \cdot  \rm { \left|0,001000000\ \frac {\rm{m}}{\rm{mm}}\right|}	 
\mbox{\I rørets ydre diameter [m]}
\end{equation}
\begin{equation}
\label{EES Eqn:15}
d_{i} = 7   \   \left[ \rm mm \right] \cdot  \rm { \left|0,001000000\ \frac {\rm{m}}{\rm{mm}}\right|}	 
\mbox{\I Rørets indre diameter [m]}
\end{equation}
\begin{equation}
\label{EES Eqn:16}
s_{q} = 25   \   \left[ \rm mm \right] \cdot  \rm { \left|0,001000000\ \frac {\rm{m}}{\rm{mm}}\right|}	 
\mbox{\I centerafstand mellem to rør lodret over hinanden [m]}
\end{equation}
\begin{equation}
\label{EES Eqn:17}
s_{r} = 18   \   \left[ \rm mm \right] \cdot  \rm { \left|0,001000000\ \frac {\rm{m}}{\rm{mm}}\right|}	 
\mbox{\I centerafstand mellem to rør diagonalt ved siden af hinanden [m] (s$_{r}$ = b$_{ribber}$ ved kun 1 rørrække eller ved flugtende rør) }
\end{equation}
\begin{equation}
\label{EES Eqn:18}
rør\$ = \SC{Copper} 	 
\mbox{\I Materiale rørene er lavet af }
\end{equation}

\vspace{0.10in}
\noindent
\rm ribber
\begin{equation}
\label{EES Eqn:19}
h_{ribber} = 200   \   \left[ \rm mm \right] \cdot  \rm { \left|0,001000000\ \frac {\rm{m}}{\rm{mm}}\right|}	 
\mbox{\I højden af én ribbe [m]}
\end{equation}
\begin{equation}
\label{EES Eqn:20}
b_{ribber} = 37   \   \left[ \rm mm \right] \cdot  \rm { \left|0,001000000\ \frac {\rm{m}}{\rm{mm}}\right|}	 
\mbox{\I bredden af én ribbe [m]}
\end{equation}
\begin{equation}
\label{EES Eqn:21}
N_{ribber} = 71	 
\mbox{\I antal ribber på varmeveksler}
\end{equation}
\begin{equation}
\label{EES Eqn:22}
\V{tykkelse} _{Ribber} = 0,1   \   \left[ \rm mm \right] \cdot  \rm { \left|0,001000000\ \frac {\rm{m}}{\rm{mm}}\right|}	 
\mbox{\I Tykkelse af ribber [m]}
\end{equation}
\begin{equation}
\label{EES Eqn:23}
Ribbe\$ = \SC{Aluminum}	 
\mbox{\I Materiale ribberne er lavet af }
\end{equation}

\vspace{0.10in}
\noindent
\rm S. 425 - 4. udgave

\vspace{0.10in}
\noindent
\rm De formler der ikke bruges skal kommenteres ud.

\vspace{0.10in}
\noindent
\rm for cirkulære ribber

\vspace{0.10in}
\noindent
\rm rektangulære ribber = flugtende rør eller ved 1 rørrække
\begin{equation}
\label{EES Eqn:24}
\beta = 1,27 \cdot  \sqrt{  \frac { \left( s_{q}/2 \right) }{  \left( s_{r}/2 \right)  }- 0,3  }	 
\mbox{\I sekskantede ribber = forsatte rør }
\end{equation}

\vspace{0.10in}
\noindent
\rm Grundlagen der värmeübertragung s. 103

\vspace{0.10in}
\noindent
\rm $^{Ribberørrækketal}$  $^{}$      1     $^{}$      2     $^{}$      3     $^{}$   $>$= 4  $^{}$\newline
	$_{;;;;;;;;;;;;;;;;;;;;;;;;;;;;;;;;;;;;;;;;;;;;;;\newline}$
	$^{}$ Flugtende     m    $^{}$   0.53   $^{}$   0.61    $^{}$   0.67   $^{}$   0.68   $^{}$\newline
	$^{}$                     C    $^{}$   0.37  $^{}$   0.295   $^{}$   0.22   $^{}$   0.16   $^{}$\newline
	$_{;;;;;;;;;;;;;;;;;;;;;;;;;;;;;;;;;;;;;;;;;;;;;;\newline}$
	$^{}$ Forsatte        m   $^{}$   0.53   $^{}$   0.56    $^{}$   0.60   $^{}$   0.63   $^{}$\newline
     $^{}$                      C   $^{}$   0.37   $^{}$   0.39    $^{}$   0.41   $^{}$   0.43   $^{}$\newline
	$_{;;;;;;;;;;;;;;;;;;;;;;;;;;;;;;;;;;;;;;;;;;;;;;\newline}$

\begin{equation}
\label{EES Eqn:25}
m = 0,60		 
\mbox{\I Aflæst værdi for m}
\end{equation}
\begin{equation}
\label{EES Eqn:26}
C = 0,41	 
\mbox{\I Aflæst værdi for C}
\end{equation}

\vspace{0.10in}
\noindent
\rm Andet om luften
\begin{equation}
\label{EES Eqn:27}
c_{L} = 3   \   \left[ \rm m/s \right]	 
\mbox{\I strømningshastighed for luft ved mindste tværsnit gennem varmeveksler [m/s]}
\end{equation}
\begin{equation}
\label{EES Eqn:28}
p_{L} = 1   \   \left[ \rm bar \right] 	 
\mbox{\I Lufttrykket [bar]}
\end{equation}

\vspace{0.10in}
\noindent
\rm $_{;;;;;;;;;;;;;;;;;;;;;;;;;;;;;;;;;;;;;;;;;;;;;;;;;;;;;;;;;;;;}$

\vspace{0.10in}
\noindent
\rm Kompressorens datablad
\begin{equation}
\label{EES Eqn:29}
\V{disp} _{k} = 5,08   \   \left[ \rm cm^{3} \right] \cdot  \rm { \left|1,\times 10^{ \cdot 6}\ \frac {\rm{m^{3}}}{\rm{cm^{3}}}\right|}	 
\mbox{\I displacement i Kompressor [m$^{3}$]}
\end{equation}
\begin{equation}
\label{EES Eqn:30}
\V{rpm}  = 4000   \   \left[ \rm 1/min \right] \cdot  \rm { \left|0,016666667\ \frac {\rm{1/s}}{\rm{1/min}}\right|}	 
\mbox{\I omdrejningshastighed af kompressoren [s$^{-1}$]}
\end{equation}
\begin{equation}
\label{EES Eqn:31}
\eta_{V} = 0,68	 		 
\mbox{\I Excel dokument om kompressor }
\end{equation}

\vspace{0.10in}
\noindent
\rm Der skal ikke indtastes mere data efter det her----------------------------------------------------------------------------------------------------------------------------------------------
\begin{equation}
\label{EES Eqn:32}
q_{Vs} = \V{disp} _{k}\cdot \V{rpm	}  
\mbox{\I slagvolumensstrøm af kompressor [m$^{3}$/s]}
\end{equation}
\begin{equation}
\label{EES Eqn:33}
q_{VR} = q_{Vs}\cdot \eta_{V	} 
\mbox{\I volumenstrømmen af kølemidlet før kompressor [m$^{3}$/s]}
\end{equation}
\begin{equation}
\label{EES Eqn:34}
c_{R;2} = q_{VR}/S_{i	} 
\mbox{\I Strømningshastigheden af kølemidlet før kompressor }
\end{equation}
\begin{equation}
\label{EES Eqn:35}
q_{mR} = q_{VR} \cdot  \rho_{R;2} 	 
\mbox{\I massestrøm af kølemidlet [kg/s]}
\end{equation}
\begin{equation}
\label{EES Eqn:36}
\rho_{R;2} = \density \left(\F{R}\$;\mbox{\ T}=t_{R;2};\mbox{\ x}=1 \right) 	 
\mbox{\I Densitet af kølemidlet før kompressoren (x = dampmængden mellem 0 og 1) [kg/m$^{3}$]}
\end{equation}



\vspace{0.10in}
\noindent
\rm $\nu$ er massestrømmen af kølemidlet bestemt!

\vspace{0.10in}
\noindent
\rm $_{;;;;;;;;;;;;;;;;;;;;;;;;;;;;;;;;;;;;;;;}$

\vspace{0.10in}
\noindent
\rm $_{;;;;;;;;;;;;;;;;;;;;;;;;;;;;;;;;;;;;;;;;;;;;;;;;;;;;;;;;;;;;;;;;;;;;;;;;;;;;;;;;;;;;;;;;;;;;;;;;;;;;;;;;;;;;;;;;;;;;;;\newline}$
 \newline
1 - Beregn den logaritmiske temperaturdifferens, ${\delta t}$$_{m}$\newline
 \newline
      1.1 - til dette benyttes formlen for middeltemperaturdifferens fra Aages noter om lamelvarmevekslere. [formel (9.69) i termodynamik 3. udgave]
\begin{equation}
\label{EES Eqn:37}
{\delta t}_{mkryds} = \epsilon \cdot  {\delta t}_{m		} 
\mbox{\I (9.70) termodynamik 3. udgave }
\end{equation}
\begin{equation}
\label{EES Eqn:38}
{\Delta t}_{m} = \frac { \left(  \left( t_{L;2} - t_{R;1} \right)  -  \left( t_{L;1} - t_{R;2} \right)  \right) }{  \left( \ln{ \left( \frac {t_{L;2} - t_{R;1}}{ t_{L;1} - t_{R;2} } \right) } \right) 	 } 
\mbox{\I (9.69) termodynamik 3. udgave }
\end{equation}

\vspace{0.10in}
\noindent
\rm \newline
$_{;;;;;;;;;;;;;;;;;;;;;;;;;;;;;;;;;;;;;;;;;;;;;;;;;;;;;;;;;;;;;;;;;;;;;;;;;;;;;;;;;;;;;;;;;;;;;;;;;;;;;;;;;;;;;;;;;;;;;;;\newline}$
 \newline
2 - Beregn den totale Varmeovergangsmodstand, SigmaR\newline
 \newline
$_{;;;;;;;;;;;;;;;;;;;;;;;;;;;;;;;;;;;;;;;;;;;;;;;;;;;;;;;;;;;;;;;;;;;;;;;;;;;;;;;;;;;;;;;;;;;;;;;;;;;;;;;;;;;;;;;;;;;;;\newline}$
 \newline
2.1 - Beregn Varmeovergangsmodstand, R$_{ou}$, ved konvektion udvendigt p\aa lameller og rør\newline
 \newline
     2.1.1 - bestem varmeovergangstal p\aa lameller og rør til luft, $\alpha$$_{u}$, ud fra Th. E. Schmidts modelligning for Nusselts tal p\aa ribberørbundt\newline
 \newline
            2.1.1.1 - beregn Nusselts tal ved brug af 2. del af modelligningen
\begin{equation}
\label{EES Eqn:39}
\V{Nuu}  = C \cdot   \left( \V{Re} _{u} \right) ^{m} \cdot   \left( A_{t}/A_{0} \right) ^{m-1} \cdot  \V{Pr} _{u}^{0,33}		 
\mbox{\I formel for Nusselts tal udvendigt på varmeveksler }
\end{equation}
\begin{equation}
\label{EES Eqn:40}
\V{Re} _{u} = c_{L} \cdot  d_{u}/\V{nu		} 
\mbox{\I Reynolds tal udvendigt på varmeveksler }
\end{equation}
\begin{equation}
\label{EES Eqn:41}
\nu = \kinematicviscosity \left(\F{Air}_{ha};\mbox{\ T }= t_{Lm};\mbox{\ P }= p_{L} \right) 	 
\mbox{\I Kinematisk viskositet for den omkringliggende luft [m$^{2}$/s]}
\end{equation}
\begin{equation}
\label{EES Eqn:42}
A_{t} = A_{0} +  \left( b_{ribber}\cdot N_{serie}\cdot h_{ribber} -  \left( d_{u}/2 \right) ^{2}\cdot \pi\cdot N_{rør} \right) \cdot 2\cdot N_{ribber	} 
\mbox{\I Total udvendigt overfladeareal af rør og ribber [m$^{2}$]}
\end{equation}
\begin{equation}
\label{EES Eqn:43}
A_{0} = d_{u} \cdot  \pi \cdot  L_{rør} -  \left( d_{u} \cdot  \pi \cdot  \V{tykkelse} _{ribber} \cdot  N_{ribber} \right) 	 
\mbox{\I Total udvendigt overfladeareal af rør mellem ribber [m$^{2}$]}
\end{equation}
\begin{equation}
\label{EES Eqn:44}
\V{Pr} _{u} = \prandtl \left(\F{Air}_{ha};\mbox{\ T }= t_{Lm};\mbox{\ P }= p_{L} \right) 		 
\mbox{\I Prandtls tal udvendigt på varmeveksler }
\end{equation}

\vspace{0.10in}
\noindent
\rm $\nu$ er der bestemt en værdi for Nusselts tal!

\vspace{0.10in}
\noindent
\rm $_{;;;;;;;;;;;;;;;;;;;;;;;;;;;;;;;;;;;;}$

\vspace{0.10in}
\noindent
\rm 2.1.1.2 - beregn $\alpha$$_{u}$ ud fra nusselt tal og 1. del af modelligningen
\begin{equation}
\label{EES Eqn:45}
\V{Nuu}  = \alpha_{u} \cdot  d_{u}/\lambda_{L	} 
\mbox{\I Generel formel for Nusselts tal benyttes til at bestemme alpha$_{u}$}
\end{equation}
\begin{equation}
\label{EES Eqn:46}
\lambda_{L} = \conductivity \left(\F{Air}_{ha};\mbox{\ T }= t_{Lm};\mbox{\ P }= p_{L} \right) 	 
\mbox{\I Varmekonduktivitet for luften [W/(m*K)]}
\end{equation}

\vspace{0.10in}
\noindent
\rm $\nu$ er varmeovergangstallet udvendigt p\aa rørene/ribber $\alpha$$_{u}$ bestemt!

\vspace{0.10in}
\noindent
\rm $_{;;;;;;;;;;;;;;;;;;;;;;;;;;;;;;;;;;;;}$

\vspace{0.10in}
\noindent
\rm 2.1.2 - Bestem det ydre overfladeareal, A$_{u}$ (denne er allerede udregnet som A$_{t}$ i afsnit 2.1.1.1)
\begin{equation}
\label{EES Eqn:47}
A_{u} = A_{t} 
\end{equation}

\vspace{0.10in}
\noindent
\rm $_{;;;;;;;;;;;;;;;;;;;;;;;;;;;;;;;;;;;;;;;;;;;;;;;;}$

\vspace{0.10in}
\noindent
\rm 2.1.3 - Bestem finnevirkningsgraden, $\eta$$_{R}$\newline
                  2.1.3.1 - Dette gøres som beskrevet i Danvak Grundbog s. 424 version 3.

\vspace{0.10in}
\noindent
\rm for en varmeveksler med forsatte rør antages der at finnevirkningsgraden kan beregnes som angivet p\aa Figur 14.10 i Danvak
\begin{equation}
\label{EES Eqn:48}
\eta_{R} = \frac {\tanh{ \left( L \cdot  \psi \right) }}{  \left( L \cdot  \psi \right)  } 
\end{equation}
\begin{equation}
\label{EES Eqn:49}
L = l_{ribbe} \cdot  \sqrt{  \frac { \left( 2 \cdot  \alpha_{u} \right) }{  \left( lampda_{Ribber} \cdot  tykkelse_{Ribber} \right)  }  } 	 
\mbox{\I L er en enhedsløs størrelse [ligning (14.7) Danvak] }
\end{equation}
\begin{equation}
\label{EES Eqn:50}
l_{ribbe} =  \left( s_{q}/2 \right)  -  \left( d_{u}/2 \right) 		 
\mbox{\I ribbehøjden regnet ud fra figur 14.10 Danvak }
\end{equation}
\begin{equation}
\label{EES Eqn:51}
\V{lampda} _{Ribber}=\conductivity \left(\F{Ribbe}\$;\mbox{\ T}=t_{R;2} \right)      	 
\mbox{\I Varmekonduktivitet for ribber, her ses på materialet for ribberne }
\end{equation}
\begin{equation}
\label{EES Eqn:52}
\psi = 1 + 0,35 \cdot  \beta \cdot  \ln{ \left( \frac {s_{q}/2}{ d_{u}/2} \right) }		 
\mbox{\I Dimensionsløs størrelse fra (14.10) i Danvak }
\end{equation}

\vspace{0.10in}
\noindent
\rm $\beta$ st\aar ved de indtastede variabler

\vspace{0.10in}
\noindent
\rm $_{;;;;;;;;;;;;;;;;;;;;;;;;;;;;;;;;;;;;;;;;;}$

\vspace{0.10in}
\noindent
\rm $\nu$ kan punkt 2.1 bestemmes (Beregn varmeovergangsmodstand, R$_{ou}$, ved konvektion udvendigt p\aa lameller og rør)

\vspace{0.10in}
\noindent
\rm det gøres med formlen:
\begin{equation}
\label{EES Eqn:53}
R_{ou} = \frac {1}{  \left( \alpha_{u} \cdot  A_{u} \cdot  \eta_{R} \right)  } 
\end{equation}

\vspace{0.10in}
\noindent
\rm $\nu$ er varmeovergangsmodstanden R$_{ou}$ bestemt!

\vspace{0.10in}
\noindent
\rm $_{;;;;;;;;;;;;;;;;;;;;;;;;;;;;;;;;;;;;;;;;;}$

\vspace{0.10in}
\noindent
\rm $_{;;;;;;;;;;;;;;;;;;;;;;;;;;;;;;;;;;;;;;;;;;;;;;;;;;;;;;;;;;;;;;;;;;;;;;;;;;;;;;;;;;;;;;;;;;;;;;;;;;;;;;;;;;;;;;;;;;;;;;;;;;;;;\newline}$
$_{;;;;;;;;;;;;;;;;;;;;;;;;;;;;;;;;;;;;;;;;;;;;;;;;;;;;;;;;;;;;;;;;;;;;;;;;;;;;;;;;;;;;;;;;;;;;;;;;;;;;;;;;;;;;;;;;;;;;;;;;;;;;;;}$

\vspace{0.10in}
\noindent
\rm Beregning af indvendigt varmeovergangstal og derefter varmeovergangsmodstand ved konvektion indvendigt i røret

\vspace{0.10in}
\noindent


% ____________ Hertil er der sat lignigner ind
\rm 2.2 - Beregn Varmeovergangsmodstand, R$_{oi}$, ved konvektion indvendigt i rør\newline
 \newline
            2.2.1 - Bestem varmeovergangstallet over rørets indervæg, $\alpha$$_{i.}$ Dette gøres ud fra Bo Pierres metode [køleteknik afsnit 6]\newline
                  2.2.1.1 - Bestem nusselts tal med Bo Pierres metode
\begin{equation}
\label{EES Eqn:54}
\V{Nui}  = 0,0075 \cdot   \left( \V{Re} _{i}^{2} \cdot  K_{f} \right) ^{0,4}	 
\mbox{\I Nusselts tal indvendig ved tør operation [formel 6.29 - køleteknik]}
\end{equation}
\begin{equation}
\label{EES Eqn:55}
\V{Re} _{i} = q_{mR} \cdot  \frac {d_{i}}{  \left( S_{i} \cdot  \eta_{l} \right) 	 } 
\mbox{\I Reynolds tal indvendig}
\end{equation}
\begin{equation}
\label{EES Eqn:56}
S_{i} = \pi/4\cdot  d_{i}^{2}	 
\mbox{\I Rørets indvendige tværsnitareal [m$^{2}$]}
\end{equation}
\begin{equation}
\label{EES Eqn:57}
\eta_{l} = \viscosity \left(\F{R}\$;\mbox{\ T }= t_{R;1};\mbox{\ x }= 0 \right) 	 
\mbox{\I Dynamisk viskositet for kølemiddel i mættet væsketilstand (x=0) [kg/(m*s)]}
\end{equation}
\begin{equation}
\label{EES Eqn:58}
K_{f} = \frac {r}{  \left( L_{rør}\cdot g \right) 	 } 
\mbox{\I Kogtal}
\end{equation}
\begin{equation}
\label{EES Eqn:59}
r = \enthalpyvaporization \left(\F{R}\$;\mbox{\ T}=t_{R;1} \right) 	 
\mbox{\I Fordampningsvarme [J/kg]}
\end{equation}
\begin{equation}
\label{EES Eqn:60}
g = 9,82   \   \left[ \rm m/s^{2} \right]	 
\mbox{\I Tyngdeacceleration [m/s$^{2}$]}
\end{equation}

\vspace{0.10in}
\noindent
\rm $\nu$ er Nusselts tal for indersiden af fordamperen bestemt!

\vspace{0.10in}
\noindent
\rm $_{;;;;;;;;;;;;;;;;;;;;;;;;;;;;;;;;;;;;;;;;;;;;;;;}$

\vspace{0.10in}
\noindent
\rm 2.1.1.2 - beregn $\alpha$$_{i}$ ud fra nusselts tal og 1. del af modelligningen

\vspace{0.10in}
\noindent
\rm Ud fra Nusselts tal, kan varmeovergangstallet bestemmes med formel [6.30]
\begin{equation}
\label{EES Eqn:61}
\V{Nui}  = \alpha_{i} \cdot  d_{i}/\lambda_{Rl	} 
\mbox{\I General formel for Nusselts tal}
\end{equation}
% ___ Hertil at lignignerne sat ind. 
\begin{equation}
\label{EES Eqn:62}
\lambda_{Rl} = \conductivity \left(\F{R}\$;\mbox{\ T}=t_{R;1};\mbox{\ x }= 0 \right)   	 
\mbox{\I Varmekonduktivitet for kølemiddel i mættet væsketilstand [W/(m*K)]}
\end{equation}

\vspace{0.10in}
\noindent
\rm $\nu$ er varmeovergangstallet $\alpha$$_{i}$ bestemt!

\vspace{0.10in}
\noindent
\rm $_{;;;;;;;;;;;;;;;;;;;;;;;;;;;;;;;;;;;;;;;;;;;;;;;;;;;;;;;;;;;;;;;;;;;;;;;;;;;;;;;;;;;;;;;;;;;;;;;;;;;;;;;;;;;;;;;;;;;;}$

\vspace{0.10in}
\noindent
\rm Herefter beregnes 2.2 (Beregn Varmeovergangsmodstand, R$_{oi}$, ved konvektion indvendigt i rør)

\vspace{0.10in}
\noindent
\rm Det gøres med formlen fra noter fra Aage (Formel 9.23, termodynamik 3. udgave)
\begin{equation}
\label{EES Eqn:63}
R_{oi} = \frac {1}{  \left( \alpha_{i} \cdot  A_{i} \right) 	 } 
\mbox{\I Varmeovergangsmodstand indvendigt på røret [K/W]}
\end{equation}
\begin{equation}
\label{EES Eqn:64}
A_{i} = d_{i} \cdot  \pi \cdot  L_{rør	} 
\mbox{\I Overfladeareal indvendigt på rørene [m$^{2}$]}
\end{equation}

\vspace{0.10in}
\noindent
\rm $\nu$ er varmeovergangsmodstanden, R$_{oi}$, bestemt!

\vspace{0.10in}
\noindent
\rm $_{;;;;;;;;;;;;;;;;;;;;;;;;;;;;;;;;;;;;;;;;;;;;;;;;;;;;;;;;;;;;;;;;;;;;;;;;;;;;;;;;;;;;;;;;;;;;;;;;;;;;;;;;;;;;;;;;;;;;;;;;;;;;;;\newline}$
$_{;;;;;;;;;;;;;;;;;;;;;;;;;;;;;;;;;;;;;;;;;;;;;;;;;;;;;;;;;;;;;;;;;;;;;;;;;;;;;;;;;;;;;;;;;;;;;;;;;;;;;;;;;;;;;;;;;;;;;;;;;;;;;;;}$

\vspace{0.10in}
\noindent
\rm 2.3 - Beregn Varmeovergangsmodstand i rørvæg, R$_{rør}$\newline
 \newline
            2.3.1 - dette gøres med formlen for varmeledningsmodstand i rørvæg [noter om lamelvarmeveklser - eller termodynamik afsnit 9]
\begin{equation}
\label{EES Eqn:65}
R_{rør} = \frac {\ln{ \left( d_{u}/d_{i} \right) }}{  \left( \lambda_{rør} \cdot  2 \cdot  \pi \cdot  L_{rør} \right) 	 } 
\mbox{\I Varmeledningsmodstand gennem rørvæggen [K/W]	Formel 9.15, termodynamik 3. udgave}
\end{equation}
\begin{equation}
\label{EES Eqn:66}
\lambda_{rør} = \conductivity \left(\F{rør}\$;\mbox{\ T}=t_{R;1} \right)  	 
\mbox{\I Varmekonduktivitet for rør [W/(m*K)]}
\end{equation}

\vspace{0.10in}
\noindent
\rm $\nu$ er varmeovergangsmodstanden i rørvæggen ogs\aa bestemt!

\vspace{0.10in}
\noindent
\rm $_{;;;;;;;;;;;;;;;;;;;;;;;;;;;;;;;;;;;;;;;;;;;;;;;;;;;;;;;;;;;;;;;;;;;;;;;;;;;;;;;;;;;;;;;;;;;;;;;;;;;;;;;;;;;;;;;;;;;;;;;;;;;}$

\vspace{0.10in}
\noindent
\rm 2.4 - læg de tre varmeovergangsmodstande sammen

\vspace{0.10in}
\noindent
\rm $\nu$ beregnes SigmaR ved at lægge de tre modstande sammen:
\begin{equation}
\label{EES Eqn:67}
\V{SigmaR}  = R_{ou} + R_{oi} + R_{rør} 	 
\mbox{\I Totale varmeovergangsmodstand fra luft til væske [K/W]}
\end{equation}

\vspace{0.10in}
\noindent
\rm $_{;;;;;;;;;;;;;;;;;;;;;;;;;;;;;;;;;;;;;;;;;;;;;;;;;;;;;;;;;;;;;;;;;;;;;;;;;;;;;;;;;;;;;;;;;;;;;;;;;;;;;;;;;;;;;;;;;;;;;;;;;;;;\newline}$
$_{;;;;;;;;;;;;;;;;;;;;;;;;;;;;;;;;;;;;;;;;;;;;;;;;;;;;;;;;;;;;;;;;;;;;;;;;;;;;;;;;;;;;;;;;;;;;;;;;;;;;;;;;;;;;;;;;;;;;;;;;;;;;;\newline}$
$_{;;;;;;;;;;;;;;;;;;;;;;;;;;;;;;;;;;;;;;;;;;;;;;;;;;;;;;;;;;;;;;;;;;;;;;;;;;;;;;;;;;;;;;;;;;;;;;;;;;;;;;;;;;;;;;;;;;;;;;;;;;;;;}$

\vspace{0.10in}
\noindent
\rm $\nu$ hvor SigmaR og ${\delta t}$$_{mkryds}$ kendes, kan kuldeydelsen, $\phi$$_{V}$, og Varmegennemgangstallet U, bestemmes

\vspace{0.10in}
\noindent
\rm Det gøres med varmetransmissionsligningen [ligning 3 i noter om lamelvarmevekslere]:
\begin{equation}
\label{EES Eqn:68}
\phi_{V} = \frac {{\delta t}_{mkryds}}{ \V{SigmaR	} } 
\mbox{\I Ligning 2.31 i køleteknik 3. udgave}
\end{equation}
\begin{equation}
\label{EES Eqn:69}
\frac {{\delta t}_{mkryds}}{ \V{SigmaR} }= U \cdot  A_{u} \cdot  {\delta t}_{mkryds	} 
\mbox{\I Ligning 2.31 i køleteknik 3. udgave}
\end{equation}

\vspace{0.10in}
\noindent
\rm Beregning af entalpier ud fra $\phi$
\begin{equation}
\label{EES Eqn:70}
q_{mR} \cdot   \left( h_{R;1} - h_{R;2} \right)  + \phi_{V} = 0	 
\mbox{\I energibalance for kølemidlet }
\end{equation}
\begin{equation}
\label{EES Eqn:71}
h_{R;1} = \enthalpy \left(\F{R}\$;\mbox{\ T }= t_{R;1};\mbox{\ x }= 0 \right) 		 
\mbox{\I x = dampmængde, 0 = væske , 1 = damp }
\end{equation}

\vspace{0.10in}
\noindent
\rm dampmængde i kølemidlet - melder fejl hvis det hele er fordampet
\begin{equation}
\label{EES Eqn:72}
q_{mL} \cdot   \left( h_{L;1} - h_{L;2} \right)  - \phi_{V} = 0	 
\mbox{\I energibalance for luftsiden }
\end{equation}
\begin{equation}
\label{EES Eqn:73}
q_{mL} = q_{vL} \cdot  \rho_{L	} 
\mbox{\I massestrøm af luft [kg/s]}
\end{equation}
\begin{equation}
\label{EES Eqn:74}
\rho_{L} = \density \left(\F{Air}_{ha};\mbox{\ T }= t_{Lm};\mbox{\ P }= p_{L} \right) 	 
\mbox{\I luftens densitet [kg/m$^{3}$]}
\end{equation}
\begin{equation}
\label{EES Eqn:75}
q_{vL} = c_{L} \cdot  A_{luftstrøm	} 
\mbox{\I volumenstrøm af luft [m$^{3}$/s]}
\end{equation}
\begin{equation}
\label{EES Eqn:76}
A_{luftstrøm} = L_{1rør} \cdot  h_{ribber} -  \left( d_{u}\cdot L_{1rør} \cdot   \left( N_{rør}/N_{rækker} \right)  \right)  
\mbox{\I tværsnitsarealet luften passerer gennem [m$^{2}$]}
\end{equation}
\begin{equation}
\label{EES Eqn:77}
h_{L;1} = \enthalpy \left(\F{Air}_{ha};\mbox{\ T }= t_{L;1};\mbox{\ P }= p_{L} \right) 	 
\mbox{\I entalpi for luft før fordamper [J/kg]}
\end{equation}
\begin{equation}
\label{EES Eqn:78}
t_{L;3} = \temperature \left(\F{Air}_{ha};\mbox{\ P }= p_{L};\mbox{\ h }= h_{L;2} \right) 	 
\mbox{\I temperatur på luft efter fordamper [K] }
\end{equation}

\vspace{0.10in}
\noindent
\rm anbefalet temperatur
\begin{equation}
\label{EES Eqn:79}
t_{LC;4} = \frac {t_{LC;2} + t_{LC;3}}{ 2	 } 
\mbox{\I Gennesmsnittet af temperatur gæt og beregning }
\end{equation}
\begin{equation}
\label{EES Eqn:80}
t_{LC;1} = \mbox{ConvertTemp}{ \left( K;\ C;\ t_{L;1} \right) } 
\end{equation}
\begin{equation}
\label{EES Eqn:81}
t_{LC;2} = \mbox{ConvertTemp}{ \left( K;\ C;\ t_{L;2} \right) } 
\end{equation}
\begin{equation}
\label{EES Eqn:82}
t_{LC;3} = \mbox{ConvertTemp}{ \left( K;\ C;\ t_{L;3} \right) } 
\end{equation}
\begin{equation}
\label{EES Eqn:83}
t_{L2gæt} = t_{lC;2} 
\end{equation}
\begin{equation}
\label{EES Eqn:84}
t_{L2beregnet} = t_{lC;3} 
\end{equation}
\begin{equation}
\label{EES Eqn:85}
t_{L2kvalbud} = t_{lC;4} 
\end{equation}
\begin{equation}
\label{EES Eqn:86}
t_{Lfør} = t_{LC;1} 
\end{equation}
\begin{equation}
\label{EES Eqn:87}
\phi_{1} = \phi_{V} 
\end{equation}
\begin{equation}
\label{EES Eqn:88}
\phi_{2} = \phi_{ønsk} 
\end{equation}
\begin{equation}
\label{EES Eqn:89}
U_{2} = U 
\end{equation}

\end{document}
