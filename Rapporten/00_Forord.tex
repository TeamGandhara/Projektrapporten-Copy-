\documentclass[../Hovedrapport.tex]{subfiles}
\begin{document}
%------------------------------------------------------------------------------
\chapter*{Forord}       % Forord nummereres ikke som kapitel
    \label{chap:forord}
%------------------------------------------------------------------------------
Denne rapport er udarbejdet af de studerende i gruppe 3, på 4. semester på maskinteknikuddannelsen ved \textit{Ingeniørhøjskolen Aarhus Universitet}. \textit{Energitekniske systemer} er det overordnede tema for projektet.

Forudsætningerne for at læse og forstå rapporten er, et vist kendskab til termodynamikkens grundlæggende principper og værktøjer; heraf forstås begreber som varmestrømme, entalpi, konduktivitet m.m. samt opstillingen af kontrolflader og energibalancer. Derudover forventes et grundlæggende kendskab til programmering i softwaren \textit{NI LabVIEW 18}. 

Der rettes stor tak til projektgruppens vejleder, Aage Birkkjær Lauritsen, for sin inspirerende vejledning og konstruktive kritik i forbindelse med projektfaget og termodynamik. Derudover skal der rettes stor tak til underviser Mikkel Bo Nielsen for sin vejledning i forbindelse med usikkerhedsberegninger og LabVIEW programmet i Instrumenteringsfaget. Ligledes skal der rettes en stor tak til Jesper Jensen og resten prototypeværkstedet, der har assisteret ved samling af køleanlægget og rådgivning omkring denne.

Rapporten består af fire overordnede dele; én hovedrapport, én forsøgsrapport, én appendiksrapport og én bilagsrapport, med passende referencer fra hovedrapporten til sidstnævnte. Hovedrapporten er yderligere opdelt i otte særskilte kapitler. Afsnittene i de enkelte kapitler startes med at blive nummeret efter deres respektive kapitler, dvs. afsnittene i kapitel to vil hedde; 2.1, 2.2, 2.3 osv.

Der vil igennem rapporten fremtræde kildehenvisninger, og disse vil være samlet i en litteraturliste bagerst i hovedrapporten. Der er i rapporten anvendt kildehenvisning efter \textit{Harvardmetoden}, hvilket medfører at der i teksten refereres til en kilde med [Efternavn, År]. Denne henvisning fører til kildelisten, hvor bøger er angivet med forfatter, titel, udgave og forlag, mens internetsider er angivet med forfatter, titel og dato. Henvisning til ligningskilder er undtaget denne metode; foruden at de følger Harvardmetoden i litteraturlisten, vil der ved benyttelse af eksterne ligninger refereres til bognavn og sidetal fortløbende med deres anvendelse.

Figurer og tabeller er nummereret i henhold til det aktuelle kapitel, som de er placeret i, dvs. den første figur i kapitel 7 har nummer 7.1, den anden, nummer 7.2 osv. Forklarende tekst til figurer og tabeller findes under de givne figurer og tabeller. 

Der er udarbejdet en evaluering af processen og tidsplan over projektforløbet, som er vedhæftet i appendiks.  
\clearpage
\end{document}
%-------------------------------------------------------------------------------