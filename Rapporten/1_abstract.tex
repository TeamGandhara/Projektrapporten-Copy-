\documentclass[../Hovedrapport.tex]{subfiles}
\begin{document}
\vspace{-30pt}
%------------------------------------------------------------------------------
\chapter*{Abstract (Alle)}       % Resume nummereres ikke som kapitel
    \label{chap:resume}
%------------------------------------------------------------------------------
% Problemet / Redegørelse:
The purpose of this paper is the calculation and construction of a refrigerator capable of maintaining a constant temperature of \SI{4}{\celsius} when exposed to a heat load consisting of 40 bottles, containing \SI{300}{mL} at \SI{21}{\celsius} of water each. The pivotal point of the project is to select the thermodynamical components, such as a compressor, heat exchangers and an expansion valve of appropriate capacity to satisfy said need.

% Anvendte metoder / Analyse:
Initially, the refrigerating plant and its components are explained. The methodology to solve said set of problems is twofold; analytical calculations based on thermodynamical principles are utilized in order to calculate a theoretical capacity for the thermodynamical components. The analytical calculations are succeeded by practical trial runs on the completed refrigerator prototype. Hence, practical measurements of temperature and pressure are analyzed and used to calculate the capacity of the components in the plant. The gathering of measurements of temperature and pressure are achieved by the development of a suitable custom software in \textit{NI LabVIEW 18}.

% Diskussion / konklusion:
Subsequently, the calculations based on the two different methods are compared and the results are discussed. The comparisons show that the theoretical and practical cooling power only differ by a small margin which is considered as satisfactory. The results of the trial run show that the temperature of the refrigerator is maintained at $\SI{4}{\celsius} \pm \SI{1,5}{\celsius}$. These results suggest, that the selected components have an appropriate capacity for the described need.

\clearpage
%-------------------------------------------------------------------------------
\end{document}