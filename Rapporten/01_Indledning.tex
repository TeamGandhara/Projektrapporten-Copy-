\documentclass[../Hovedrapport.tex]{subfiles}
\begin{document}
%------------------------------------------------------------------------------
\chapter{Indledning (Alle)}       % Indledning nummereres ikke som kapitel
    \label{chap:indledning}
    \vspace{-30pt}
%------------------------------------------------------------------------------
Denne opgave vil beskrive beregningen og konstruktionen af et køleskab til en bar. På baggrund af indledende beregninger på forskellige komponenter i køleanlægget, vil en prototype blive udarbejdet i praksis med henblik på at foretage forsøg, som tager afsæt i tænkte scenarier i en bar. Resultaterne fra disse forsøg ønskes derefter sammenholdt med de teoretiske beregninger. Herved kan eventuelle forbedringspunkter blive belyst og tilpasset med henblik på en eventuel opfølgende prototype. 

Som et led i konstruktionen, er der i Instrumenteringsfaget udarbejdet et usikkerhedsbudget for relevante målestørrelser og udviklet et styringsprogram til køleskabet. Programmet er bl.a. med til at sikre et mere jævnt temperaturforløb i køleskabet, logning af diverse målinger af temperaturene og trykkene i anlægget og har desuden også til formål at kalibrere sensorene til disse målestørrelser, således at de indledende beregninger kan valideres.

Komponenter som ekspansionsventil, fordamper og kondensator er blevet udvalgt på baggrund af en ønsket kompressor. Hertil er en kuldeydelse og kondenseringsydelse blevet bestemt ud fra kompressorens valgte driftsforhold, hvorpå fordamper og kondensator er blevet valgt på baggrund af at kunne opfylde disse ydelser. Kompressoren og de teoretiske beregninger tager udgangspunkt i den maksimalt ønskede kuldeydelse, som køleskabet er bestemt til at kunne operere ved. Det anvendte kølemiddel i systemet er pålagt til at være R134a. 

Parametre som varmegennemgangstal, kuldeydelse og andre tilførte varmestrømme under belastningsforhold beregnes endvidere ud fra data målt i forsøgene. De dertilhørende beregninger er lavet på baggrund af de driftsforhold, som ovenstående forsøg foregår under. Forsøgene har dermed givet et indblik i, hvorledes de teoretiske beregninger stemmer overens med de førnævnte parametre fra forsøgsresultaterne under testforholdene. Slutteligt diskuteres resultaterne og de faktorer, som ligger til grund for afvigelserne.
\clearpage
%-------------------------------------------------------------------------------
\end{document}