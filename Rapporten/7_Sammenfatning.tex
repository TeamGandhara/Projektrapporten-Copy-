\documentclass[../Hovedrapport.tex]{subfiles}
    \begin{document}
%-------------------------------------------------------------------------
\chapter{Sammenfatning (Alle)}
    \label{chap:Perspektivering}
    \vspace{-30pt}
%--------------------------------------------------------------------------
\section{Perspektivering til virkeligheden}
I følgende afsnit vil der blive foretaget en perspektivering til køleskabe, som allerede befinder sig på markedet. 

Der er en del forskellige versioner af køleskabe på markedet, og forbrugerens behov er afgørende for, hvilken type der vælges. Nogen forbrugere har et lille volumenbehov, mens andre har et behov for at køleskabet er stort nok til at kunne opbevare mad- og drikkevarer til en uges forbrug for en stor familie. For nogen forbrugere er indkøbsprisen afgørende, mens det for andre er vigtigere, at modellen er kostøkonomisk på længere sigt, hvilket bl.a. kan gøres ved at vælge en energieffektiv model.\\
Når køleskabet i projektet skal sammenlignes med andre standardkøleskabe, er der en del forskelle. Normale køleskabe er ofte drevet af en AC-kompressor, hvilket bl.a. skyldes, at indkøbsprisen er lavere ved denne type end ved en DC-kompressor, hvor sidstnævnte bliver benyttet i projektet. Der er dog flere producenter, der er begyndt at benytte omdrejningsregulerede kompressorer i køleskabe. Virksomheden \textit{LG} har bl.a. lavet køleskabe, som er omdrejningsregulerede, i form af deres \textit{inverter compressor}. I denne kompressor bliver vekselstrømmen omdannet til jævnstrøm og dette muliggør, at kompressoren kan levere præcis den kuldeydelse, der er behov for i køleskabet. Dette medfører en større energibesparelse i og med, at der er færre start og stop af kompressoren, hvilket resulterer i en energibesparelse på op til $40\%$. Desuden hævder \textit{LG}, at denne type kompressor er mindre støjende end en almindelig AC-kompressor samt at denne kan holde en mere jævn temperatur  \citep{learneng}.  \\
Foruden den omdrejningsregulerede kompressor er der også benyttet en kondensator med tvungen luftstrømning. Dette benyttes også ved visse moderne køleskabe, hvilket bl.a. er med til at give mere dybde i køleskabet og dermed mere plads til mad og drikke, hvor der stadig er den samme kondenseringsydelse i kondensatoren.\\
Slutteligt påføres fordamperen også en tvungen luftstrømning. En ventileret køling er især benyttet i større køleskabsanordninger, da dette, ligesom kompressoren, giver en mere jævn temperatur i køleskabet. Dette skyldes, at den varme luft stiger opad og den kolde nedad, hvilket skaber stor forskel på temperaturen i køleskabets top og bund. Ved påsætning af en ventilator på fordamperen kan det undgås, at der er halvlunkne madvarer på den ene hylde og frosne madvarer på den anden hylde, da luften herved bliver blandet rundt i rummet \citep{whiteaway}.
%----------------------------------------------------------------------
    \newpage
\section{Diskussion}
% En lille indledning:
Projektforløbet bygger på en række beslutninger, antagelser samt iterative beregningsprocesser, som er truffet og foretaget undervejs gennem hele forløbet. Disse beslutninger og antagelser afspejler sig i sidste ende direkte i de resultater, som opnås heraf. Der kan herefter reflekteres over, hvilke influensparametre, der kunne justeres næste gang et køleanlæg skal designes. Disse overvejelser vil diskuteres i det følgende.

% Alt med kompressoren:
Kompressoren er en essentiel del af køleanlægget. Det konstruerede køleskab er blevet testet under forskellige belastningsscenarier. Testresultaterne viser, at kapaciteten af den udvalgte kompressor til anlægget til tider overgår eller undergår det nødvendige behov.

Derudover har det vist sig, at kompressorens kapacitet overstiger det nødvendige behov, når belastningen på den ikke er tilstrækkelig stor. Dette observeres, når køleskabets indetemperatur er sammenfaldende med den ønskede temperatur, som skal holdes konstant. Herved er det nødvendige kuldebehov til opretholdelse af den konstante temperatur mindre end \SI{222}{\watt}, der er den mindste kuldeydelse, som kompressoren kan sikre, uden at slukke. Det er ikke muligt at afhjælpe denne problemstilling med en regulering alene, da kompressoren ikke kan operere med et lavere omdrejningstal end det der producerer \SI{222}{\watt} ved de beskrevne forhold. En regulering af kompressoren er stadig formålstjenstligt, idet den bidrager til et mindre elforbrug, hvis betingelsen for en eventuel regulering eksisterer, hvilket vil sige, at der er et kuldebehov mellem \SI{222}{\watt} og \SI{334}{\watt}.

% Skal nedenstående være mere uspecifik. 'INDSÆTTES' ER RETTET
I andre situationer er kapaciteten for den udvalgte kompressor modsat ikke tilstrækkelig. Dette kan eksempelvis være ved indsættelse af en betydelig mængde drikkevarer ved stuetemperatur, som overstiger antallet af flasker anvendt i forsøgene, i køleskabet sideløbende med, at døren ofte åbnes. 

Ved en opskalering af køleskabets fysiske dimensioner vil belastning stige på flere områder, da overfladearealet og den indre volumen forøges. Der vil indledningsvist være et større tab igennem væggene grundet det større overfladeareal, mens en forøget volumen muliggør, at en større mængde varm luft kan befinde sig i kølesskabet. Dette vil samtidig resultere i, at færre start og stop af kompressoren vil finde sted, idet dens anvendelsesinterval vil ligge tættere på \SI{222}{\watt}. Ligeledes vil en opskalering også resultere i en større døråbning, hvorved en større varmestrøm kan blive tilført, når køleskabsdøren åbnes. 

Hvis en mindre kompressor i stedet var blevet valgt, er det muligt, at omdrejningsreguleringen havde været mere funktionel. Problemet med en mindre kompressor er, at den ikke vil kunne overholde kravsspecifikationerne med henhold til serveringstemperaturen for drikkevarerne. I dette tilfælde, hvor køleskabet er tiltænkt at skulle stå i en bar, er kompressorens kapacitet ofte passende, idet den jævnligt bliver hårdt belastet. Derfor vurderes det konstruerede køleskab at have en køleskabskapacitet, som er for stor til at omdrejningsreguleringen af kompressoren vil fungere til et køleskab i et almindeligt hjem.  

Kompressorens kondenseringsydelse er indledningsvis bestemt i Coolselector ud fra aktuelle nøgletemperaturer for anlægget \citep{Coolselector}. Men da Coolselecter ikke tager højde for, at kompressoren er luftkølet, har dette bevirket, at en teoretisk kondenseringsydelse er beregnet, hvor der tages højde for køletabet. Den beregnede kondenseringsydelse bygger på en antagelse om, at kompressorens køletab består af 20 \% af den tilførte el-effekt. Men kompressorens sande køletab er en funktion af dens temperatur i forhold til omgivelsernes temperatur, hvorfor denne bør kendes i virkeligheden. Dette bidrager til, at der vil opnås kendskab til kompressorens sande kondenseringsydelse ved en vilkårlig temperatur.

% HUSK HUSK HUSK:
% Herved vil den valgte kompressor ikke kunne opfylde kravspecifikationerne på maksimal 7 graders serveringstemperatur, grundet den større varmetilførsel gennem døråbning.

% Kondensatoren og fordamperen:
Foruden kompressoren er anlæggets varmevekslere også af interesse. Det har her været en begrænsende faktor, at der kun har været lamelvarmevekslere i udvalgte geometriske dimensioner tilgængelig. Dette har influeret beregningsproceduren til udvælgelsen af lamelvarmevekslerne, idet en eksisterende tilgængelig varmeveksler er valgt og dernæst efterberegnet i forhold til den ønskede kulde- og kondenseringsydelse. 

Det vurderes, at beregningsproceduren kunne være foretaget omvendt, idet en varmeveksler kan dimensioneres på baggrund af et aktuelt behov. Dette havde været brugbart, hvis der var mulighed for at benytte lamelvarmevekslere af vilkårlige dimensioner. Dette har dog ikke været muligt, da omfanget for en sådan procedure ligger uden for de opstillede rammer for opgaven. Med den tidsmæssige horisont taget i betragtning har det bedre kunnet svare sig at efterregne en tilgængelig varmeveksler.

Derudover er der til beregning af kapaciteten af varmevekslerne anvendt en række forsimplinger. Forsimplingerne har været nødvendige for at muliggøre udregningerne, idet beregningen af de reelle processer i varmevekslerne er for komplicerede. Heriblandt kan udeladelsen af \textit{Lockheed-Martinelli metoden}, til bestemmelse af kondensatorens indvendige varmeovergangstal, nævnes. Denne metode er ikke benyttet, idet dampkvaliteten af kølemidlet, som funktion af rørlængden af kondensatoren, ikke kendes.

% Ventilen:
Foruden dette har den valgte ekspansionsventil med dysestørrelse \textit{0X} en kapacitet, som overstiger det nødvendige behov. Dette giver anledning til, at den ikke lukker en jævn strøm af kølemiddel igennem, men veksler imellem at lukke for meget og for lidt igennem. Dette skyldes, at kuldebehovet er for lille i forhold til den valgte ventil. En ekspansionsventil i en passende størrelse har dog ikke været muligt at anskaffe grundet en for lang leveringstid. For den valgte ventil udnyttes blot 46\% af dens kapacitet, når anlægget er belastet maksimalt.

% Forsøgene og sensorer:
Ved udførelsen af forsøgene er kompressorens tilførte el-effekt ikke målt. Denne burde have været målt med eksempelvis en velkalibreret wattmåler, hvorefter den målte el-effekt kunne være sammenlignet med den teoretisk beregnede el-effekt. For de anvendte sensorer i anlægget, til måling af tryk og temperatur, antages det, at samtlige sensorer har den samme usikkerhed. I virkeligheden burde et usikkerhedsbudget for de enkelte sensorer være opstillet, hvilket ville have bidraget til en større præcision og troværdighed af sensorernes målinger. 

% Regulering og styring: 
Endelig er der uvished omkring kompressorens reelle omdrejningstal, når køleskabet er i drift, idet det ikke har været muligt at måle eller beregne den. Der sås derfor tvivl omkring, hvorvidt reguleringen af kompressoren har haft en indflydelse. Derudover er den benyttede PID-funktion i \textit{LabVIEW} ikke designet til langsomme ændringer af en given målestørrelse, men derimod bratte ændringer. LabVIEW har en såkaldt PID-autotuning funktion, som er en tilpasning af den normale PID-regulering. Denne funktion ville have været mere fordelagtig i dette tilfælde, hvor langsomme temperaturændringer kan forekomme. 

% Konstruktion og prototype 2:
Der er plads til optimering med hensyn til konstruktionen af anlægget og placeringen af dets komponenter. Kondensatoren og kompressoren ønskes placeret så højt som muligt på køleskabet. Dette bevirker, at den opadgående varmestrøm fra disse komponenter ikke opvarmer køleskabet og dermed modregner dens nedkøling. Derudover kan kondensatoren med fordel blæse henover kompressoren, som vil bidrage til en nedkølingen af denne, når den er blevet tilstrækkelig varm. Derudover er prototypen ikke tæt i sprækker og hjørner, hvorigennem varme kan trænge ind i køleskabet. Ydermere kan køleanlæggets effektivitet forbedres ved at ændre fordamperens placering. Hvis denne komponent i stedet var placeret i toppen af køleskabet ville det mindske behovet for en blæser, idet den nedkølede luft naturligt ville bevæge sig til bunden af køleskabet og dermed skabe en cirkulation. I det nuværende tilfælde vil fordamperens placering nederst i køleskabet formentlig resultere i en lagdeling af luften, hvis der ikke benyttes en blæser. Der bør dog tages forbehold for, at denne anordning af kompressoren og fordamperen bidrager til en øget instabilitet af køleskabet, hvorfor der bør tages hensyn til dette.

Ydermere er mange barkøleskabe et såkaldte "Display Køleskabe", hvilket betyder at fronten er lavet af glas, mens der også er indsat i lys inde i køleskabet med henblik på at fremvise drikkevarer for kunderne. Varmekonduktiviteten for opskummet polysstyren er dog betydeligt mindre end for glas, hvilket bidrager til at kompressoren i dette projekt ikke skal fjerne lige så meget varme og dermed er for stor. Desuden anvendes der heller ikke lys i køleskabet i dette projekt, hvilket også har et varmebidrag omend kun nogle få watt.
 
Det observeres desuden, at det tager et kort øjeblik for blæseren til fordamperen at komme i stilstand, når køleskabsdøren åbnes. Dette bidrager til, at den kolde luft blæses ud, indtil blæseren kommer i stilstand. Derfor bør blæseren have en større nedbremsning. Endelig ønskes længden af sugegasledningen udenfor køleskabet forkortet. Dette vil bevirke, at det kolde kølemiddel ikke opvarmes af omgivelserne, inden det kommer ind i køleskabet.

For yderligere indsigt i en tænkt Prototype 2 henvises der til appendiks \ref{sec:proto1} og \ref{sec:proto2}. 

\section{Konklusion}
I denne rapport er beregningerne og konstruktionen af et køleskab blevet beskrevet. Det er lykkedes at udforme et aggregat med en kuldeydelse så tilstrækkelig stor, at en betydelig indsættelse af drikkevarer hurtigt kan blive nedkølet uden at have nævneværdig indflydelse på de varer, som allerede befinder sig i køleskabet. I praksis er dette blevet undersøgt ved et forsøg, hvor 40 flasker vand á 0,30 L er blevet nedkølet fra stuetemperatur samt ved et forsøg med indsættelse af lunkne vandflasker i køleskabet, mens drikkevarer allerede befandt sig heri. Slutteligt er køleanlægget også i stand til at nedkøle og udligne temperaturdiffensen for køleskabsluften i løbet af 1 minut efter 30 sekunders åbning af køleskabsdøren. I forsøgene overstiger temperaturen for de flasker, som allerede befinder sig i køleskabet inden belastning påføres, ikke \SI{7}{\celsius} og opfylder dermed det opstillede krav.  

Det er muligt at ønske en køleskabstemperatur via software udviklet i forbindelse med projektet. Det er dog ikke lykkedes at opnå et jævnt temperaturforløb, som ellers var formålet med at vælge en DC-kompressor. Lufttemperaturen ligger her på \SI{4}{\celsius} \pm \SI{1,5}{\celsius}, men bliver justeret via et reguleringssytem, som var et af projektets hovedformål. Temperaturintervallet overskrider altså det, som er blevet opstillet i kravsspecifikationen. Dette er en konsekvens af, at kompressorens minimale kuldeydelse på \SI{222}{\watt} overstiger det nødvendige kuldebehov under stationære forhold. Derudover har størrelsen af dysen i ekspansionsventilen også bidraget til de ujævne temperaturforløb.
\end{document}