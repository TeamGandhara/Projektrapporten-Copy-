\documentclass[../Hovedrapport.tex]{subfiles}
    \begin{document}
%------------------------------------------------------------------------
\section{Procesbeskrivelse (Alle)}
    \label{sec:batman}
%------------------------------------------------------------------------
Der er igennem projektet gjort nogle erfaringer. Disse erfaringer er blevet diskuteret i gruppen, og der er bred enighed fra alle medlemmer omkring det der vil blive beskrevet i denne sektion. Procesbeskrivelsen opdeles i 2 sektioner. 'Faglige erfaringer' og 'samarbejdets erfaringer'. \\ 
\textbf{Faglige erfaringer:} \\
Inden projektets begyndelse var det gruppens forventning, at få en dybere forståelse for hvordan et kølesystem fungere, samt hvordan der regnes på diverse elementer heraf. Dette er også vurderet til at være indfriet, da der igennem forløbet har været et stort fokus på at udføre udregninger på bla. fordamper og kondensatoren.   
Alle gruppemedlemmer føler at de har fået en dybere forståelse for hvordan et køleteknisk system fungere, samt hvordan de individuelle dele influere kredsprocessen. 

Afslutningsvis har gruppen også erfaret vigtighede af gode illustrationer til at hjælpe formidlingen af komplekse forklaringer. 

\textbf{Samarbejdets erfaringer} \\
I starten af projektet blev der udarbejdet en tidsplan. Denne tidsplan blev inddraget i flere sammenhænge i starten af projektet. Dette betød at der i denne perode var et godt overblik over hvor langt processen var i forløbet. Tidsplanen mistede dog det fokus, som blev givet den, og den blev næsten ikke inddraget den sidste måned. Dette skete da gruppen blev opdelt, for bla. at nå diverse deadlines i form af milepæl 2 samt portifolio opgaver i INS. Derfor mistede vi også overblikket til tider i løbet af den sidste slutspurt. 
Hele denne oplevelse har været med til at cementere vigtigheden af at bevare overblikket, og dertil er en tidsplan et uvurderlig værktøj.

Igennem projektet har det virket godt at have mindre løbende deadlines. Dette har været i form af bla. milepæl 1 og 2, de 4 portifolioopgaver samt mindre deadlines selv opstillet af gruppen. Dette har været med til at bevare overblikket, som også er beskrevet ovenstående til at være yderst vigtigt. 

Overordnet er samarbejdet i gruppen forløbet problemfrit. Dette hænger sandsynligtvis også sammen med at gruppen arbejdede sammen i det foregående semester. Derfor kender de forskellige medlemmer hinandens styrker og svagheder, hvilket har været med til at højne effektiviteten.

Alt i alt er alle medlemmer tilfredse med forløbet, og der vurderes et stort udbytte, både fagligt men også i forhold til hvordan fremtidigt projektarbejde skal forløbe. 


\end{document}